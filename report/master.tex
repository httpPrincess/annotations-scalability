\documentclass[a4paper,10pt]{article}
\usepackage[utf8]{inputenc}
\usepackage[english]{babel}
\usepackage{graphicx}
\usepackage{listings}

\usepackage{xcolor}

\colorlet{punct}{red!60!black}
\definecolor{background}{HTML}{EEEEEE}
\definecolor{delim}{RGB}{20,105,176}
\colorlet{numb}{magenta!60!black}

\lstdefinelanguage{json}{
    basicstyle=\normalfont\ttfamily,
    numbers=left,
    numberstyle=\scriptsize,
    stepnumber=1,
    numbersep=8pt,
    showstringspaces=false,
    breaklines=true,
    frame=lines,
    backgroundcolor=\color{background},
    literate=
     *{0}{{{\color{numb}0}}}{1}
      {1}{{{\color{numb}1}}}{1}
      {2}{{{\color{numb}2}}}{1}
      {3}{{{\color{numb}3}}}{1}
      {4}{{{\color{numb}4}}}{1}
      {5}{{{\color{numb}5}}}{1}
      {6}{{{\color{numb}6}}}{1}
      {7}{{{\color{numb}7}}}{1}
      {8}{{{\color{numb}8}}}{1}
      {9}{{{\color{numb}9}}}{1}
      {:}{{{\color{punct}{:}}}}{1}
      {,}{{{\color{punct}{,}}}}{1}
      {\{}{{{\color{delim}{\{}}}}{1}
      {\}}{{{\color{delim}{\}}}}}{1}
      {[}{{{\color{delim}{[}}}}{1}
      {]}{{{\color{delim}{]}}}}{1},
}

  
%opening
\title{neo4j performance and scalability evaluation}
\author{Jedrzej Rybicki WP9.2 Lead}

\begin{document}

\maketitle

\section{Introduction}
EUDAT WP9 is a activity which looks into new technologies, evaluate their 
applicability for building new services or substitute existing ones. In one of 
its subtasks (WP9.2) it looks into graph databases. It was already shown that 
graph databases have the potential to provide different view on the stored 
research data, enabling new type of search queries, and extracting insights 
through graph-based analytics. To become new service or substitute existing
one, however, graph databases have to provide production-ready performance and
scalability. To this end, EUDAT Technical Committee charged WP9.2 with a task 
of evaluating performance and scalability of neo4j graph database.

The neo4j graph database is widely used and it does not make much sense to 
conduct generic scalability tests of this technology in EUDAT. It is much
more sensible to do some EUDAT-specific tests. Optimally comparing the 
performance of neo4j with a technology currently used in EUDAT. Unfortunately,
there were not much performance tests done in EUDAT so far. Thus we have
selected one candidate service (B2NOTE) which uses popular noSQL database (mongodb)
as our benchmark. 

\section{Methodology/Setup}
B2NOTE service is implemented to enable semantic annotations of the documents 
stored in the EUDAT domain. It uses W3C format for storing 
annotations\footnote{https://www.w3.org/TR/annotation-model/} in mongodb. W3C 
web annotation data format is pretty simple: Each annotation is a relation 
between a body (e.g. EUDAT data object), and target (e.g. metadata describing 
that object). Basic annotation model is shown on Listing~\ref{lst:anno}.

\begin{lstlisting}[language=json,frame=single,caption=Basic W3C annotation data model,label=lst:anno]
{
  "@context": "http://www.w3.org/ns/anno.jsonld",
  "id": "http://example.org/anno2",
  "type": "Annotation",
  "body": {
    "id": "http://example.org/analysis1.mp3",
    "format": "audio/mpeg",
    "language": "fr"
  },
  "target": {
    "id": "http://example.gov/patent1.pdf",
    "format": "application/pdf",
    "language": ["en", "ar"],
    "textDirection": "ltr",
    "processingLanguage": "en"
  }
}
\end{lstlisting}

It is important to notice that both target and body have unique identifiers. 
These are crucial from the user perspective. It is to expect that the users will 
be interested to view list of all annotations for given body id (i.e. all 
metadata descriptions for given data object). But also a ``reverse'' lookup 
producing all the dataobjects with specific tag (i.e. a retrieval by target id) 
embodies important functionality. From this requirements it becomes clear what 
kind of queries the backend database has to deal with. Thus, we defined three 
metrics for the database:
\begin{itemize}
 \item creation times (creation of new, non-existing annotation),
 \item annotation retrieval by target id,
 \item annotation retrieval by body id.
\end{itemize}

To obtain meaningful results it is important to minimize the number of ``moving 
parts'' and reduce the testing environment to parts which are absolutely 
necessary. In particular we were not interested in the performance of the B2NOTE 
web interface or the performance penalty caused by integration with other EUDAT 
services. Therefore we have written a simple program in Python with methods 
for generating annotations with unique body and target id, and for retrieval
of the data. The methods use simple interfaces to access different database
stores: mongo and neo4j. We made sure that we are using the sample mongo API 
as B2NOTE is currently using ($pymongo==3.3.0$). 

To enable easy reproducibility of the conducted tests, a docker-based environment
was prepared. Both technology providers (mongo and neo4j) offer official docker
images for their databases. We created a docker image with our testing program
and prepared a docker-composed-based testing environment. Given a system
with running docker and docker-compose, starting tests is a matter of merely 
issuing one command like: 
\begin{verbatim}
 $ docker-compose run tester --name experiment1 
\end{verbatim}

Also docker images for processing of the results and visualizing them are 
provided. All the source code and documentation is stored in 
github\footnote{https://github.com/httpPrincess/annotations-scalability} 
enabling verification and repetition of the tests. In fact we plan to 
redo similar tests for different EUDAT-inspired use cases. 

\section{Results}
The tests are defined by tree parameters:
\begin{itemize}
 \item engine: database engine (currently mongo and neo4j),
 \item runs: number of rounds, 
 \item reps: number of repetitions in each round.
\end{itemize}

The tests were divided into rounds and in each round all the above database operations were 
conducted in the given order. In each round firstly $reps$ number of records were 
created, subsequently random (with repetition) $reps$ were retrieved by 
specifying existing $target.jsonld_id$ and afterwards $reps$ random annotations 
were fetched by $body.jsonld_id$. We measured time of each activity, that is 
complete time to create records, time to retrieve all $reps$ record by target id 
(or body id). Three time measurements were made in each round. Please note, that 
no records were removed i.e., for given $reps = 1000$, database grown in each round 
by new 1000 record. 


\begin{figure}
\centering
 \includegraphics[width=.7\textwidth]{fig/mongo-ret-scalability} 
 \caption{mongo retrieval scalability (db is growing \emph{reps} records and \emph{reps} random records are retrieved in each round)} \label{fig:mongo-ret}
\end{figure}


\begin{figure}
\centering
 \includegraphics[width=.7\textwidth]{fig/neo-ret-scalability}
 \caption{neo4j retrieval scalability (db is growing $reps$ records and $reps$ random records are retrieved in each round)} \label{fig:neo-ret}
\end{figure}

On Fig.~\ref{fig:mongo-ret} and Fig.~\ref{fig:neo-ret} we depicted the retrieval scalability 
of each database. For that we conducted three experiments with different values of $reps$, each 
had 10 rounds. Fig~\ref{fig:mongo-ret} shows that the performance of mongodb is 
dramatically decreasing with the increasing size of the database. Also the absolute values 
achieved by mongo are not very well, to retrieve $10000$ random annotations from a database 
with $90 000$ documents, more than one minute is required. 

The retrieval times for the same amount of data from the neo4j database of the same size is 
much smaller as can be seen on Fig.~\ref{fig:neo-ret}. Also the scalability of neo4j is 
much better, neo4j produces constant answer times regardless of the size of the database. 


\begin{figure}
\centering
 \includegraphics[width=.7\textwidth]{fig/creation}
 \caption{Comparison of creation scalability ($reps$ records are added in each round)} \label{fig:creation}
\end{figure}

The situation is a little bit different for creation times. We depicted them on 
Fig.~\ref{fig:creation}. For smaller values of $reps$ neo4j outperforms mongo but 
with $reps = 5000$ mongo is faster. We also conducted the tests for higher values of 
$reps$ (not depicted for the sake of clarity) and mongo was also faster. neo4j 
also displays high variance in the creation times. Perhaps further investigations
are required there. 

\begin{figure}
\centering
 \includegraphics[width=.7\textwidth]{fig/neovsmongo}
 \caption{Scalability mongo vs. neo4j} \label{fig:mongovsneo}
\end{figure}

To provide direct comparison of read performance and scalability we conducted 
a test with fixed $reps = 5000$ and results are depicted on Fig.~\ref{fig:mongovsneo}.
Neo4j is not only faster on in the whole spectrum of database size (rounds) but 
also scales very well with the growth of the database what is not true for mongo.


\section{Summary}
Performance results should never be the final result of any project. They, rather, constitute
a starting point for further performance tuning. It is true for both mongo and neo4j. The database
management engines offer many possibles for parameter tuning. 

\end{document}

